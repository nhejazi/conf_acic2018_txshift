%%%%%%%%%%%%%%%%%%%%%%%%%%%%%%%%%%%%%%%%%
% baposter Landscape Poster
% LaTeX Template
% Version 1.0 (11/06/13)
%
% baposter Class Created by:
% Brian Amberg (baposter@brian-amberg.de)
%
% License:
% CC BY-NC-SA 3.0 (http://creativecommons.org/licenses/by-nc-sa/3.0/)
%
%%%%%%%%%%%%%%%%%%%%%%%%%%%%%%%%%%%%%%%%%

%----------------------------------------------------------------------------------------
% PACKAGES AND OTHER DOCUMENT CONFIGURATIONS
%----------------------------------------------------------------------------------------

\documentclass[landscape,a0paper,fontscale=0.285]{baposter} % Adjust font scale/size

\usepackage{graphicx} % Required for including images
\graphicspath{{figs/}} % Directory in which figures are stored

\usepackage{amsmath} % For typesetting math
\usepackage{amssymb} % Adds new symbols to be used in math mode
\usepackage{mathtools}

\usepackage[numbers]{natbib}

\usepackage{booktabs} % Top and bottom rules for tables
\usepackage{enumitem} % Used to reduce itemize/enumerate spacing
\usepackage{palatino} % Use the Palatino font
\usepackage[font=small,labelfont=bf]{caption} % Required for specifying captions to tables and figures

\usepackage{multicol} % Required for multiple columns
\setlength{\columnsep}{1.5em} % Slightly increase the space between columns
\setlength{\columnseprule}{0mm} % No horizontal rule between columns

\newcommand{\compresslist}{ % Define a command to reduce spacing within itemize enumerate environments, this is used right after \begin{itemize} or \begin{enumerate}
\setlength{\itemsep}{1pt}
\setlength{\parskip}{0pt}
\setlength{\parsep}{0pt}
}

\definecolor{lightblue}{rgb}{0.145,0.6666,1} % Defines color of content box headers

\begin{document}

\begin{poster} {
headerborder=closed, % Adds a border around the header of content boxes
colspacing=1em, % Column spacing
bgColorOne=white, % Background color for the gradient on the left side of the poster
bgColorTwo=white, % Background color for the gradient on the right side of the poster
borderColor=lightblue, % Border color
headerColorOne=black, % Background color for the header in the content boxes (left side)
headerColorTwo=lightblue, % Background color for the header in the content boxes (right side)
headerFontColor=white, % Text color for the header text in the content boxes
boxColorOne=white, % Background color of the content boxes
textborder=roundedleft, % Format of the border around content boxes, can be: none, bars, coils, triangles, rectangle, rounded, roundedsmall, roundedright or faded
eyecatcher=true, % Set to false for ignoring the left logo in the title and move the title left
headerheight=0.1\textheight, % Height of the header
headershape=roundedright, % Specify the rounded corner in the content box headers, can be: rectangle, small-rounded, roundedright, roundedleft or rounded
headerfont=\Large\bf\textsc, % Large, bold and sans serif font in the headers of content boxes
%textfont={\setlength{\parindent}{1.5em}}, % Uncomment for paragraph indentation
linewidth=2pt % Width of the border lines around content boxes
}
%----------------------------------------------------------------------------------------
% TITLE SECTION
%----------------------------------------------------------------------------------------
%
{\includegraphics[height=6.5em]{logo_cal.jpg}} % First university/lab logo on the left
{\bf\textit{\LARGE Robust Nonparametric Inference for Stochastic Interventions
    Under Multi-Stage Sampling}\vspace{0.01em}} % Poster title
{\textbf{Nima S.~Hejazi, Mark J.~van der Laan, Holly E.~Janes, Peter B.~Gilbert,
    and David C.~Benkeser} \\ \textit{Group in Biostatistics \& Department of
    Statistics, University of California, Berkeley}} % Author names and institution
{\includegraphics[height=6.5em]{logo_sph.jpg}} % Second university/lab logo on the right

%-------------------------------------------------------------------------------
% OVERVIEW
%-------------------------------------------------------------------------------

\headerbox{Overview \& Motivations}{name=overview,column=0,row=0}{

\begin{enumerate}\compresslist
\setlength\itemsep{0.75em}
\item We consider the problem of efficiently estimating survival prognosis under
  a data structure complicated by the presence of immortal time bias.
\item The matter of efficient estimation under a bias induced by time-dependent
  risks presents a novel challenge that received surprisingly meager attention
  in the literature.
\item We compare parametric and nonparametric estimators of survival, including
  variations of the Cox proportional hazards model and the Kaplan-Meier
  estimator, evaluating the efficiency of each in the estimation of
  the multiple survival processes that occur under this data-generating process.
\item We are given survival times for patients with a single primary melanoma,
  and some of the patients develop a second primary melanoma before dying.
\end{enumerate}

%\vspace{0.3em} % When there are two boxes, some whitespace may need to be added
               % if the one on the right has more content
}

%-------------------------------------------------------------------------------
% INTRODUCTION
%-------------------------------------------------------------------------------

\headerbox{Introduction \& Data}
{name=introduction,column=1,row=0,bottomaligned=overview}{

\begin{itemize}\compresslist
\setlength\itemsep{0.75em}
\item Question of interest: \textbf{How does the second melanoma change the
    survival prognosis of the patients?}
\item In order to prepare for a real data analysis, we simulate a data structure
  that matches what we expect --- that is, the data-generating process is the
  the Cox proportional hazards model.
\item Survival time $T$: time before the actual death of the patient,
\item Time until second melanoma appears: $U$,
\item Baseline hazard in absence of second melanoma: $\lambda_0(t)$,
\item Time-varying covariate: $Z(t) = I(t > U)$.
\item Constant baseline hazard $\lambda_0 = \lambda$.
\item A second melanoma doubles the hazard.
\end{itemize}
}

%-------------------------------------------------------------------------------
% Methodology cont.
%-------------------------------------------------------------------------------

\headerbox{Methodology II}{name=results,column=2,span=2,row=0}{


\vspace{-0.35em}
\begin{itemize}
\item The second approach is non-parametric and uses Kaplan-Meier's estimator
  defined as
\end{itemize}
\begin{equation*}
\widehat{S}(t) = \prod_{i : t(i) < t} \left(1 - \frac{d_i}{n_i}\right), \quad
t\geq 0,
\end{equation*}
where $d_i$ and $n_i$ are the respective numbers of death and individual at
risks at the ordered time $t^{(i)}, \ i = 1, \ldots, n$.
\begin{itemize}
\item Youlden et al. \cite{youlden2016ten} only uses patients for whom no
  occurrence of a second melanoma is observed, in the estimation of $S_1$ and
  ignores the other patients, which causes a bias.
\item Jewell corrects their estimator by including all the patients in the
  study.
\item The ones that were excluded by Youlden et al. \cite{youlden2016ten} still
  contain information about $\lambda_1$: those are censored observations at time
  $U$.
\end{itemize}
}

%-------------------------------------------------------------------------------
% REFERENCES
%-------------------------------------------------------------------------------

\headerbox{Principal References}{name=references,column=2,above=bottom}{
\renewcommand{\section}[2]{\vskip 0.05em} % remove "References" section title
\nocite{*} % Insert publications even if they are not cited in the poster
\tiny{ % Reduce the font size in this block
\setlength{\bibsep}{1pt}
\bibliographystyle{unsrt}
\bibliography{2018_acic}\compresslist
}
}

%-------------------------------------------------------------------------------
% CONTACT
%-------------------------------------------------------------------------------

\headerbox{Contact Information}{name=ack,column=3,aligned=references,above=bottom}{
% This block is as tall as the references block
  \textbf{N.~Hejazi}: \textsc{nhejazi@berkeley.edu} \\
  \textbf{M.J..~van der Laan}: \textsc{laan@berkeley.edu} \\
  \textbf{D.~C.~Benkeser}: \textsc{benkeser@emory.edu}
}

%-------------------------------------------------------------------------------
% CONCLUSION
%-------------------------------------------------------------------------------

\headerbox{Results \& Discussion}
{name=conclusion,column=2,span=2,row=0,below=results,above=references}{

\vspace{1em}
\begin{multicols}{2}

\begin{center}
\vspace*{-0.45cm}
\includegraphics[scale=0.5]{s1_estim_compare}
\vspace{-1.5em}
\captionof{figure}{Average performance of estimators for $S_1$ for a sample of
 size $n = 1000$, over about $300$ simulations.}
\end{center}

\vspace{3em}

\begin{itemize}
  \setlength\itemsep{0.2em}
  \item The Kaplan-Meier estimator proposed by Youlden displays obvious bias.
  \item The estimates of the survival curve produced by Cox regression and the
    Kaplan-Meier estimator with the Jewell correction show no such bias.
  \item Under the Cox model, Cox regression will outperform other estimators ---
    it draws upon information across both subject groups over all time points.
  \item The Kaplan-Meier estimator exhibits a slight divergence from the truth
    in the right tail due to a well-studied finite-sample bias caused by
    censored observations.
  \item We display results for $n = 1000$ since this sample size is closest to
    that from the observational medical study we analyze.
\end{itemize}

\end{multicols}

}

%-------------------------------------------------------------------------------
% METHODS
%-------------------------------------------------------------------------------

\headerbox{Methodology I}{name=method,column=0,span=2,below=overview,bottomaligned=references}{
% This block's bottom aligns with the bottom of the conclusion block

\begin{itemize}
\item Time origin for all sujects: date of their first or index primary melanoma
  (PM).
\item Two hazard functions of essential interest:
\item $\lambda_1(t)$ --- hazard of an individual, alive at time $t$ who has only
  experienced one PM.
\item $\lambda_2(t)$ --- hazard of
an individual, alive at time $t$ who has experienced more than one PM.
\item Data: of $n = n_1 + n_2$ subjects.
\item The first $n_1$ only experience one PM before death at the observed time
  $t_i$. The other $n_2$ experience a second PM at the observed time $u_j$ and
  then die at observed time $t_j$. It is possible to consider censored
  observations for both sets of subjects but we do not discuss this here for the
  sake of notation.
\item We compare three approaches of this problem, namely, the Cox proportional
  hazards model, the method presented in Youlden et al. \cite{youlden2016ten}
  and its correction by Jewell.
\end{itemize}

The basic proportional hazards model is a semi-parametric model for the hazard
function defined by
\begin{equation}
\lambda\left(t; Z = z\right) = \lambda_0(t) \exp\left(\beta^Tz\right), \quad
t\geq 0.
\end{equation}
where $\lambda_0(\cdot)$ is the baseline hazard function is estimated
non-parametrically, while $\beta$ is the vector of regression coefficients and
is estimated parametrically using Cox's partial likelihood.
}

%-------------------------------------------------------------------------------

\end{poster}
\end{document}

